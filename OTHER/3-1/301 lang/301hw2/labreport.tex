
\documentclass[11pt]{article}

\marginparwidth 0.2in 
\oddsidemargin 0.1in 
\evensidemargin 0.1in 
\marginparsep 0.1in
\topmargin 0.1in 
\textwidth 6.5in \textheight 8 in

\usepackage{hyperref}  
\usepackage{listings}
\usepackage{array}
\usepackage{multirow}
\usepackage{attachfile}
\usepackage{lscape}
\usepackage{graphicx}
\lstset {tabsize=4}

\begin{document}

\author{İbrahim Burak Tanrıkulu, 21827852}
\title{BBM301 Programming Languages\\Tail recursion and Iterative functions in Scheme}
\maketitle

\section*{PART A: Converting recursive functions to tail recursive ones}

{\bf 1- Finding length of a list with tail recursion:}\\~\\
step 0:\hspace{0,5 cm} (length '(1, 2, 3, 4, 5))\\
step 1:\hspace{0,5 cm} (length\_helper '(1, 2, 3, 4, 5) 0)\\
step 2:\hspace{0,5 cm} (length\_helper '(2, 3, 4, 5) (+ 1 0))\\
step 3:\hspace{0,5 cm} (length\_helper '(3, 4, 5) (+ 1 1))\\
step 4:\hspace{0,5 cm} (length\_helper '(4, 5) (+ 1 2)\\
step 5:\hspace{0,5 cm} (length\_helper '(5) (+ 1 3))\\
step 6:\hspace{0,5 cm} (length\_helper '() (+ 1 4))\\
step 7:\hspace{0,5 cm} lst is null, return 5.\\~\\
We just store current\_length variable and increase it at each stage. We don't store return\\
address and parameters of each recursive function. Thus we gained time and space.
\\~\\
{\bf 2- sum-of-squares tail recursion:}\\~\\
{\bf a)}
\begin{lstlisting}
(define (sum-of-squares n)
(letrec (
        (sum-of-squares-helper (lambda (n sum)
                   (if (= n 0)
                     sum
                     (sum-of-squares-helper (- n 1) (+ sum (* n n)))
                   ))
        ))
(sum-of-squares-helper n 0)
))
\end{lstlisting}
{\bf b)}\\
In this part, i will use "SoS" abbreviation for sum-of-squares and SoSh for sum-of-squares-helper.
\begin{lstlisting}
		recursive              tail recursive
step 0:	 (SoS 5)               (SoS 5)
step 1:	 -(SoS 4)              (SoSh 5 0)
step 2:	 --(SoS 3)             (SoSh 4 25)
step 3:	 ---(SoS 2)            (SoSh 3 41)
step 4:	 ----(SoS 1)           (SoSh 2 50)
step 5:	 -----(SoS 0)          (SoSh 1 54)
step 6:	 -----0                (SoSh 0 55)
step 7:	 ----1                 55
step 8:	 ---5
step 9:	 --14
step 10: -30
step 11: 55
\end{lstlisting}
As you can see, recursive is slower than tail recursive. Also, recursive uses more stack space.
\\~\\
{\bf 3- sum-of-factorials-of-elements:}\\~\\
{\bf a)}\\
\begin{lstlisting}
(define sum-of-factorials-of-elements
    (lambda (lst)
      (if (null? lst)
        0
        (+ (factorial (car lst)) (sum-of-factorials-of-elements (cdr lst))))))
\end{lstlisting}
{\bf b)}\\
\begin{lstlisting}
(define (sum-of-factorials-of-elements lst)
(letrec(
       (sofoeh (lambda (lst sum)
               (if (null? lst)
                   sum
                  (sofoeh (cdr lst) (+ sum (factorial (car lst))))
       ))
  ))
  (sofoeh lst 0)
))
\end{lstlisting}
\newpage~\\
{\bf c)}\\
In this part, i will use "sofoe" abbreviation for sum-of-factorials-of-elements.
\begin{lstlisting}
		recursive steps
step 0:	 (sofoe '(3 2 5 1 4))
step 1:	 (+ 3! (sofoe '(2 5 1 4)))
step 2:	 (+ 3! (+ 2! (sofoe '(5 1 4))))
step 3:	 (+ 3! (+ 2! (+ 5! (sofoe '(1 4)))))
step 4:	 (+ 3! (+ 2! (+ 5! (+ 1! (sofoe '(4))))))
step 5:	 (+ 3! (+ 2! (+ 5! (+ 1! (+ 4! (sofoe '()))))))
step 6:	 (+ 3! (+ 2! (+ 5! (+ 1! (+ 4! 0)))))
step 11:    153
                tail recursion steps
step 0:	(sofoe '(3 2 5 1 4))
step 1:	(sofoeh '(3 2 5 1 4) 0)
step 2:	(sofoeh '(2 5 1 4) 6)
step 3:	(sofoeh '(5 1 4) 8)
step 4:	(sofoeh '(1 4) 128)
step 5:	(sofoeh '(4) 129)
step 6:	(sofoeh '() 153)
step 7:	153
\end{lstlisting}
\section*{PART B: Writing iterative functions}

{\bf 1- sum-of-squares iterative:}\\
\begin{lstlisting}
(define (sum-of-squares n)
(do ( (i 1 (+ i 1)) (sum-of-squares 0) )
    ((> i n) 
        sum-of-squares)
        (set! sum-of-squares (+ sum-of-squares (* i i)))))
\end{lstlisting}
~\\
{\bf 2- sum-of-factorials-of-elements:}\\
\begin{lstlisting}
(define (sofoe lst)
(do ( (mylist lst (cdr mylist)) (sofoe 0))
    ((null? mylist) 
        sofoe)
        (set! sofoe (+ sofoe (factorial (car mylist))))))
\end{lstlisting}
~
\section*{Comparing sum-of-squares}
I used (time (sum-of-squares 30000000)) command for getting time.\\~\\
\begin{tabular} {|c|c|c|c|}
 \hline
 & Recursive & Tail Recursive & Iterative \\
\hline
Storage & Return addresses, arguments for each call & Extra return variable & Extra loop variable \\
\hline
Time & 10.7 & 3.4 & 4.5 \\
\hline
\end{tabular}\\~\\~\\
I didn't used any reference. All these are my own code and comments. I used repl.it and \\tutorialspoint sites' scheme interpreter.
\end{document}
